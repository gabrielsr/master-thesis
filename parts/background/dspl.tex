\section{Dynamic Software Product Lines}
Researchers have investigated Dynamic Software Product Line (DSPL) a way of adapt for variations in users requirements and system environments.
% Dynamic variability (dynamic changes in the variation points and binding), context awareness, autonomous or self-adaptation are some of the necessary properties for Dynamic Software Product Line (DSPL), prescribed by Hallsteinsen et al.[0].”

DSPLs extend the concept of conventional SPLs by enabling software-variant generation at runtime. In classic SPL products can be derived from a SPL infrastructure for a specific customer individual or customer segment, in the assumption that the requirements for that customer and the execution environment will not change. In DSPLs a product can change to another configuration, in runtime, in response to a context change. To make it possible the feature model should be available at runtime. \cite{bencomo_view_2012}

DSPLs use the features models and  orthogonal variability models (OVMs) as techiniques for variaility management, to model what are valid variabilities.
