\subsubsection{Dependency Injection}

Dependency Injection is a pattern that allow for wiring together software components that was developed without the knowledge about it other.~\cite{fowler_inversion_2004}

In OO languages normally you instantiate an Object from a class using an operator (\emph{new} for Java) and a reference to this class. By this the object that is instantiating (the object client) is dependent of the referenced class (the service implementation class). In case of strong typed languages, normally one will get an exception if the referenced class is not present.

So the use of the \emph{new} operator lead to the following disadvantages:
\begin{itemize}
  \item impose compile time dependency between two classes
  \item impose runtime dependency between two classes
\end{itemize}

The basic idea of the Dependency Injection, is to have a separate object, an assembler, that wire together the components at runtime\cite{fowler_inversion_2004}. The client class refer to service using its Interface (the service interface). The assembler can use alternative ways of the \emph{new} to instantiating an object, so that the wiring between client objects and implementation service classes could be postpone to runtime.

By this is we can at runtime:
\begin{itemize}
  \item discover available implementation of a service interface
  \item decide this implementation to instantiate from available implementations
\end{itemize}

Not always this pattern is needed as not always is useful to avoid the reference to a class. But in the context of component-based adaptation it would be specially useful to the decouple client components from service components, allowing runtime reasoning about what implementation to choose.

\begin{figure}[h!]\centering

\begin{tikzpicture}
  \begin{umlpackage}[x=4,y=0]{service interface}
    \umlemptyclass[width=15ex]{IServiceA}
  \end{umlpackage}
  \begin{umlpackage}[x=4,y=-3]{service implementation}
    \umlemptyclass[width=15ex]{ServiceAImpl}
    \umlimpl{ServiceAImpl}{IServiceA}
  \end{umlpackage}
  \begin{umlpackage}[x=0,y=0]{Client}
    \umlemptyclass[width=15ex]{ClientOfServiceA}
    \umlassoc{ClientOfServiceA}{IServiceA}
  \end{umlpackage}
  \begin{umlpackage}[x=0,y=-3]{Assembler}
  \end{umlpackage}
\end{tikzpicture}

\caption{Two components}
\end{figure}

% \subsubsection{Inversion of Control Containers}
% \label{sec:iocc}
%
% It have been used as a core principle in the development of popular frameworks like Sprint and Angular.
