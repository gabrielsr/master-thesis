\section{Software Components and Architecture}

Heineman define \emph{software component} as a
\say{software element that conforms to a component model and can be independently deployed and composed without modification according to a composition standard}\cite{heineman_component-based_2001}.

Software components is a unit of composition. Software systems are build by composing different components.  Software components must conform to a component model by having contractually specified interfaces and explicit context dependencies only.\cite{szyperski_component_2002}.

A \emph{component	interface} \say{defines a set of component functional properties, that is, a set of actions that’s understood by both the interface provider (the component) and user (other components, or other software that interacts with the provider)}\cite{crnkovic_software_2011}.
A component interface has a role as a component specification and also a means for interaction between the component and its environment.
A \emph{component model} is a set of standards for a component implementation. These standards can standardize naming, interoperability, customization, composition, evolution and deployment.\cite{heineman_component-based_2001}

The \emph{component deployment} is the process that enables component integration into the system. A deployed component is registered in the system and ready to provide services\cite{crnkovic_software_2011}.

\emph{Component binding} is the process that connects different components through their interfaces and interaction channels.

Software architecture deals with the definition of components, their external behavior, and how they interact.\cite{kaur_component_2010}

Component based software engineering (CBSE) approach consists in building systems from components as reusable units and keeping component development separate from system development\cite{crnkovic_software_2011}.

CBSE is built on the following four principles\cite{crnkovic_software_2011}:
\begin{itemize}
  \item Reusability. Components, developed once, have the potential for reuse many times in different applications.
  \item Substitutability. Systems maintain correctness even when one component replaces another.
  \item Extensibility. Extensibility aims to support evolution by adding new components or evolving existing ones to extend the system’s functionality.
  \item Composability. System should supports the composition of functional properties (component binding). Composition of extra functional properties, for example composition of components’ reliability, is another possible form of composition.
\end{itemize}
