We see a growing interest in computing applications that should rely on heterogeneous computing environments, like Internet of Things (IoT).
Such applications are intended to execute in a broad range of devices with different available computing resources.
In order to handle some kind of heterogeneity, such as two possible types of graphical processors in a desktop computer, we can use simple approaches as a script at deployment-time that chooses the right software library to be copied to a folder.
These simple approaches are centralized and created at design-time. They require one specialist or team to control the entire space of variability.
However, such approaches are not scalable to highly heterogeneous environments.
In highly dynamic and heterogeneous environment it is hard to predict the computing environment at design-time, implying likely undecidability on the correct configuration for each environment at design-time.
In our work, we propose Goalp: a method that allows autonomous deployment of systems by reflecting about the goals of the system and its computing environment. By autonomous deployment, we mean that the system can find the correct set of components, for the target computing environment, without human intervention.

We evaluate our approach on the filling station advisor case study where an application advise a driver where to refuel its vehicle. Results show that using the approach it is possible to design the application with variability at requirements, architecture, and deployment, which can allow the designed application be executed in different devices. For scenarios with different environments, it was possible to plan the deployment autonomously. Additionally, the scalability of the algorithm that plan the deployment was evaluated in a simulated environment. Results show that using the approach it is possible to autonomously plan the deployment of a system with thousands of components in few seconds.
