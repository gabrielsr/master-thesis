%Conclusions(0.75p)
\section{Conclusion and future work}
\label{sec:conclusion}

% TODO relate with multi-level adaptation.
In this work, we presented Goalp, a novel approach to tackle deployment in highly heterogeneous computing environments.
Goalp allows systems deployment to heterogeneous environments, partially unknown at design-time, without requiring a system administrator.
Goalp consists in support to design a system with the needed variability to handle the heterogeneity, from requirements, through architecture, and deployment.
And in online support for solving the variability at deployment time, finding the correct set of artifacts that allows the user achieve its goals in a given target computing environment. Goalp uses a CGM to specify variability at requirements. Further, patterns are used to map components from the CGM and keep the variability are architecture level and deployment level. The novelty of our approach is that we provide a systematic way to design a system with focus in variability from requirements to deployment.
% As such, using a goal-oriented approach to deployment is expected to integrate with other approach that handle variability in another level.

Following our approach the system implemented reflects the goal-model, keeping the goals traceable to components and artifacts. Via such traceability the adequate set of artifacts is autonomously chosen achieving the target software goal in a given computing environment. Since goal models are highly abstract models, using it to drive the system adaptation, we expect to achieve a higher level of flexibility transcending the lower-level abstraction computing layers. In addition, by using context-goal models, we can handle computing resources variability. By using CGM for deployment, rework is avoided, as CGM is a model already developed in the requirements elicitation stage.

In a preliminary evaluation, we applied the Goalp approach in a case study. Further, we evaluated the scalability of the algorithm when planning in a large scenario, using a randomly generated repository and deployment requests. The results show that the algorithm is capable of coming up with a plan, in a reasonably large scenario in few seconds.

This work fits in our long-term vision of a method for design systems with variability at all stages of system design, from requirements to deployment. And a self-adaptable platform that can adapt the software deployment in order to make high-level user goals achievable. This work fits in this vision by providing the knowledge and planning part in a MAPE-K\cite{kephart_vision_2003} architecture.
We should note that this work aims at identifying a single valid deployment plan, as long as it exists. However, it is out of scope of our current work to find the best valid plan in case multiple valid plans exist. In future work, a CSP approach might integrate our deployment plan algorithm to address such issue.
For future work, we plan to: (1) extend Goalp with deployment planning for multiple nodes by including delegation as another form of variability;  (2) evolve Goalp deployment planning in a self-adaptive approach for deployment, based on MAPE-K, with addition of monitoring, analyzing, and executing capabilities; (3) evaluate Goalp in an open adaptation scenario with multiple developers providing components to the environment; and (4) evaluate self-adaptation at deployment level as a method of fault-tolerance that adapts the system deployment in response to failures in resources.
%We provide a approach that To the best of our knowledge no other approach has tackle this problem.
%We believe that this is will be useful in domain such as ubiquitous at where no specialized people interact with the system and would like it to adapt its deployment.

%The advantages of goal-driven is:
%High level of flexbility by using the system goals as a model for evaluate alternatives of adaptation.
%Easy the development by reuse the goal model.
%The advantages of using a MAS approach is the distributed nature of the system and to avoid have a sigle point of failure.
