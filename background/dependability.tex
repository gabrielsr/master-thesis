\section{Dependability}
Dependability can be defined as the ability of a system to avoid faults in its services
that (1) are more frequent or (2) more severe than acceptable. Or as the characteristic of a system to be justifiably trusted.

A common terminology used for system deviations is the following: \cite{avizienis_basic_2004}

\begin{itemize}
  \item \textbf{failure}: (or \textif{service failure}) is a perceived deviation from the correct service provided by a system.
  \item \textbf{error}: is a deviation of correct internal system state that can lead to its subsequent failure.
  \item \textbf{fault}: is the adjudged or
hypothesized cause of an error
\end{itemize}

Dependability includes the following attributes:\cite{avizienis_basic_2004}
\begin{itemize}
  \item \textbf{availability}: readiness for correct service.
  \item \textbf{reliability}: continuity of correct service.
  \item \textbf{safety}: absence of catastrophic consequences on the
user(s) and the environment.
  \item \textbf{integrity}: absence of improper system alterations.
  \item \textbf{maintainability}: ability to undergo modifications and repairs.
\end{itemize}

Many means have been developed of how to attain the attributes of dependability. These means can be classified as:

\begin{itemize}
  \item \textbf{Fault prevention} means to prevent the occurrence or introduction of faults.
  \item \textbf{Fault tolerance} means to avoid service failures in the presence of faults.
  \item \textbf{Fault removal} means to reduce the number and severity of faults.
  \item \textbf{Fault forecasting} means to estimate the present number, the future incidence, and the likely consequences of faults.
\end{itemize}


% Erica Jen defines evolvability as an entity’s ability to “alter their structure or function so as to adapt to changing circumstances”
%robustness: Feature persistence under specified and unforeseen perturbations, obtained by switching among multiple strategic options such that those changes are dynamically tolerated.


\textif{Resilient systems} \cite{laprie_dependability_2008} are expected to continuously provide justifiably be trusted services despite changes coming from the environment or from their specifications.


\section{Attain Dependability at Runtime }
To keep dependability in face of uncertainty in the deployment environment some techniques have been proposed for runtime analysis at runtime.

% TODO you need to refer to other work related to dependability at runtime: antonio filieri work, QoSA, Salehie, Sam Malek. Not only the work we develop!

Felipe et al\cite{guimaraes_framework_2013} propose a method of fault-tolerance for a scientific workflow execution in grid.

Alessandro Leite \cite{ferreira_leite_user_2014} propose a fault tolerance schema for cloud deployment based on which a fault instance in the cloud is monitored and in case of failure the instance can be restarted or terminated and them a new instance created.

Danilo et al\cite{mendonca_dependability_2015} propose a methodology for fault forecasting by which developer, at design time, annotate the goal decomposition in goal model and specify context variables. A special tool generate a formula for, given a context, evaluate the probability of achieve a goal at runtime.

% TODO Pessoa et al \cite{pessoa_dependable_2015} propose a ... with and evaluate it with focus on safety ...


 % reduce redundancy
