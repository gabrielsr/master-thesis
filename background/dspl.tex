\section{Dynamic Software Product Lines}
Researcher have investigated dynamic software product line a way of adapt to a growing need for variations in users requirements and system environments.

DSPLs extend the concept of conventional SPLs by enabling software-variant generation at runtime. In classic SPL products can be derived from a SPL infrastructure for a specific customer individual or customer segment, in the assumption that the requirements for that customer and the execution environment will not change. In DSPLs a product can change to another configuration, in runtime, in response to a context change. To make it possible the feature model should be available at runtime. \cite{bencomo_view_2012}

DSPLs use the features models and  orthogonal variability models (OVMs)
 as techiniques for vability management, to model what are valid variabilities.
