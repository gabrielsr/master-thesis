Gotel et al. \cite{gotel_analysis_1994} defines Requirements Traceability (RT) as \say{the ability to describe and follow the life of a requirement, in both a forwards and backwards direction}. (i.e from origins, through its development, to its subsequent deployment and use). The author differentiate traceability in two types. The first, pre requirements specification, refer to trace baseline changes of the requirements specification to their originating statement(s), through the process of requirements production and refinement. It refer to the requirements engineering process. The second type, post requirements specification, refers to trace from, and back to, a baseline of requirement specification, through a succession of artifacts in which they are distributed.
In this work we are interested in the second type.

%Traceability is required component of the approaval and certification process in most safety-critical systems\cite{cleland-huang_software_2014}.
%And a requirement for hight level in mature  models like CMMI \cite{} and \cite{}.

%sugests that traceability is successfully implemented in some projects within some organizations while the majority of
%projects fail to achieve effective traceability or incur excessive costs in so doing.
