\section{Problem Definition}

% is there any component model suitable to trace goals and component service.

To allow the system make decisions about its structure based on requirements and context we need a model that can correlate this 3 concepts: the system structure, the requirements and the context.

What leads us to or general question:


\setlength{\fboxsep}{10pt}
\noindent\fbox{%
    \parbox{0.95\textwidth}{%
        \textbf{Research Question 1 (RQ1):} What would be a good model of software system that
        could allow for reason about the system structure, context and trace the requirements at runtime? In another words, represent the system requirements, structure, operation context and the relationship between this elements?
    }%
}\bigskip

As this question is too difficult to answer directly we will make a proposal of solution and evaluate it. In this work we choose to represent requirements in a Goal Model based approach, the system structure from an architectural point of view and the context as a data reference resolution process.


\setlength{\fboxsep}{10pt}
\noindent\fbox{%
    \parbox{0.95\textwidth}{%
        \textbf{Research Question 2 (RQ2):} Would a model that represent system requirements at runtime as goals, system organization at a component level and a context as a data reference resolution process a model that satisfy RQ1?
    }%
}\bigskip

Yet in relation to RQ1, to develop what would be a good model we did more questions in relation to the fitness of the model for the purpose of decide on system adaptations. The following is in direction of find if a given configuration of the system is valid.

\setlength{\fboxsep}{10pt}
\noindent\fbox{%
    \parbox{0.95\textwidth}{%
        \textbf{Research Question 3 (RQ3):} How to, using the model from RQ1, verify if systems goals are achievable at runtime in face of deployment (configuration) uncertainty.
    }%
}\bigskip

Beside check the system validity in other to forecast faults we want to be able to tolerate faults. What leads to the next question:

\setlength{\fboxsep}{10pt}
\noindent\fbox{%
    \parbox{0.95\textwidth}{%
        \textbf{Research Question 4 (RQ4):} How to assure that goals are always achieved, if they are achievable, in face of components faults ?
    }%
}\bigskip

This research question is the search to insert fault-tolerance, to achieve dependability of the system in face of not dependable components. To achieve a greater level of manutenability, the faul-tolerance techniques should be portable.


\setlength{\fboxsep}{10pt}
\noindent\fbox{%
    \parbox{0.95\textwidth}{%
        \textbf{Research Question 5 (RQ5):}	Is it feasible to implement fault tolerance techniques (\textit{retry, retry on alternate resource, check-
point/restart and replication}) as portable components that plug in the runtime model of the system?
    }%
}\bigskip

We also want to asses if the model is a practical solution.

\setlength{\fboxsep}{10pt}
\noindent\fbox{%
    \parbox{0.95\textwidth}{%
        \textbf{Research Question 6 (RQ6):} The model could be used to build a middleware that allows systems build on top of it to asses itself capacity of fulfill their requirements, self-heal in case of not and tolerate faults?
    }%
}\bigskip

As a side goal, we want to build a community around the middleware tool so we can evaluate it practical value for development of systems by third party.

\setlength{\fboxsep}{10pt}
\noindent\fbox{%
    \parbox{0.95\textwidth}{%
        \textbf{Research Question 7 (RQ7):} How to engage the scientific community in a common experiment setup for self-adaptation, especially for dependability attributes, so we can compare different approaches ?
    }%
}\bigskip
%
% As it is not enough be possible, it needs to be reproducible:
%
% \setlength{\fboxsep}{10pt}
% \noindent\fbox{%
%     \parbox{0.95\textwidth}{%
%         \textbf{Research Question 8 (RQ8):} Is it feasible to create an integrated development process for self-adaptive software by bridging goal-oriented requirements engineering with architecture-based adaptation?
%     }%
% }\bigskip
