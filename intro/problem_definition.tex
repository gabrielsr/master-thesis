\section{Problem Definition}

% is there any component model suitable to trace goals and component service.



This work propose a middleware for adaptable open-systems.

The identified requirements are:
\begin{itemize}

 \item Adaptable

 The system should keep a model of itself and allow changes at runtime.

 \item Support for Open-System

 The system should have support for discover and leverage new resources

 \item Decentralized

 The system should be deployed in multiple nodes, without a singe point of failure.

 \item Evolvable

 The system could incorporate new functions and adaptiveness capabilities at runtime

 %\item Model at Runtime
 %The system should have access to a model of itself at runtime.
\end{itemize}

Our proposal built on top of multi-agent system concepts, goal-model and component-based adaptable architectures.



\setlength{\fboxsep}{10pt}
\noindent\fbox{%
    \parbox{0.95\textwidth}{%
        \textbf{Research Question 1 (RQ1):} Is there any component model suitable for trace goals to architectural components at runtime?
    }%
}\bigskip


\setlength{\fboxsep}{10pt}
\noindent\fbox{%
    \parbox{0.95\textwidth}{%
        \textbf{Research Question 1 (RQ1):} How to verify if systems goals are achievable at runtime in face of deployment (configuration) uncertainty.
    }%
}\bigskip


\setlength{\fboxsep}{10pt}
\noindent\fbox{%
    \parbox{0.95\textwidth}{%
        % every goal instance matters
        \textbf{Research Question 1 (RQ1):} How to garante that goals instances are always achieved if they are achievable ?
    }%
}\bigskip

This research question is the search for achieve dependability of the system in face of not dependable components. The system should be able to try different available strategies to achieve a goal.

\setlength{\fboxsep}{10pt}
\noindent\fbox{%
    \parbox{0.95\textwidth}{%
        \textbf{Research Question 1 (RQ1):} How to assure that goals instances are achieved if they are achievable ?
    }%
}\bigskip


- adapt to satisfy another a specific goal. How not lost any goal instance?


% até quanto? qualitative / quantitative

% first: traceability model
How can we trace runtime component-systems from goal oriented perspective


what would be a suitable model to represent the traceability between GORE and CBSA?
% Model


% component model that satisfy



% analise and act?


% Define architecture,

%

\setlength{\fboxsep}{10pt}
\noindent\fbox{%
    \parbox{0.95\textwidth}{%
        \textbf{Research Question 1 (RQ1):} How to engage the community in a common experiment setup for self-adaptation, especially for dependability attributes, so we can compare different approaches ?
    }%
}\bigskip


\setlength{\fboxsep}{10pt}
\noindent\fbox{%
    \parbox{0.95\textwidth}{%
        \textbf{Research Question 1 (RQ1):} Is it feasible to create an integrated model process for adaptive software by bridging goal-oriented requirements engineering with architecture-based adaptation?
    }%
}\bigskip


\setlength{\fboxsep}{10pt}
\noindent\fbox{%
    \parbox{0.95\textwidth}{%
        \textbf{Research Question 2 (RQ2):}	Is it feasible to implement fault tolerance techniques (\textit{retry, retry on alternate resource, check-
point/restart and replication}) as portable components that plug in the runtime model of the system?
    }%
}\bigskip


\setlength{\fboxsep}{10pt}
\noindent\fbox{%
    \parbox{0.95\textwidth}{%
        \textbf{Research Question 3 (RQ3):}	Is it feasible to develop a Goal oriented runtime platform for opportunistic open-systems?
    }%
}\bigskip

\setlength{\fboxsep}{10pt}
\noindent\fbox{%
    \parbox{0.95\textwidth}{%
        \textbf{Research Question 4 (RQ4):}	Is it feasible to develop self-adaptable strategies that are portable across applications?
    }%
}\bigskip
